\subsection{Design and operation of the second generation of interferometric gravitational wave detectors}\label{subsec:2ndgen}

%Beside generally aiming at a factor 10 overall improvement in sensitivity, and an expansion fo the sensitive frequency band towards lower frequencies, the design of the advanced detectors was aimed at making them limited by thermal noise, laser radiation pressure and laser shot noise. As a consequence, all other possible sources of noise needed to be pushed well below these main ones.

\subsubsection{Advanced LIGO}

\paragraph*{Suspesions and seismic isolation.}
Two of the subsystem that underwent the biggest upgrades in advanced LIGO as compared to its predecessor were the Seismic Isolation and Suspension subsystems. Although they are formally two separate subsystems, they work in concert to isolate the test-masses and other critical optics from ground vibrations and other macroscopic motions, and ensure that the they can move as free masses in the relevant degree of freedom above a few Hz.

In Advanced LIGO, the test masses are suspended by four-stages pendula. From top to bottom, the main suspension chain is comprised of two metal masses, and two optics of the same shape and size, the lowermost one being the test mass. The first metal mass is attached to the suspension structure by four \textit{spring blades}, which are flexible metal blades providing vertical isolation; steel wires run from the tip of the blades to the suspended mass. In a similar fashion, the second metal mass is attached to the first one, and so is the topmost optic. In the lower stage of the suspension, however, the test mass is attached to the topmost optic by four fused silica fibers; these are directly welded to the two optics in a monolithic assembly, such as to reduce mechanical losses and, consequently, thermal noise. The pendulum resonances of the four stages are distributed between 1 and 4 Hz, providing a passive suppression of motion along the optical axis of $10^-7$ at 10\.Hz, and going down as the frequency to the eighth power.

A similar chain of masses, called reaction chain, hangs parallel to the main one, and supports the sensors and actuators used for local damping and active alignment of the lowermost three masses of the main chain; this provides a quiet reference point for the control forces. The top masses of both chains are instead actuated using the suspension structure as a reference. For all stages except the lowest, the sensors/actuators are compact units that uses shadow sensors to measure the position of, and electromagnets to exert forces on, permanent magnets attached to the masses. The last stage has no local sensors, since the position of the test mass is sensed by the global interferometry; on the end test mass suspensions, the actuators consist of a patterns of electrodes deposited on the last mass of the reaction chain, referred to as reaction mass, which exert electrostatic attractive forces on the test mass when polarized. This avoids the need of attaching magnet or any other part ot the TM, thus maintaining low mechanical losses and reducing possible couplings to external fields. On the input test masses suspension, on the other end, the main laser beam is transmitted through the lowermost optics of both chains; in this case no electrodes are deposited on the reaction mass, that is instead referred to as \textit{compensation plate} for its role in the thermal compensation system, as explained in \ref{sec:presfut:ground:2gen:aLIGO:TCS}.

Each quad suspension is attached to an in-vacuum seismic isolation platforms, used for further suppression of ground vibrations and precise positioning and alignment with a larger range than allowed by the suspensions themselves. The platforms are six-axis, two-stages active and passive isolators proving more than 3 orders of magnitude isolation above 1\.Hz, and positioning capabilities with nm resolution over a range of several mm.

Similar seismic isolation platforms are used to support all the in-vacuum optics; the optics that are part of an optical cavity are installed by triple-pendula, while single-pendulum suspensions, or specialized geometries, are used to isolate less critical optics where necessary.

In both detectors, each of the in-vacuum seismic isolation platforms is installed on beams that are decoupled from the vacuum system itself via flexible bellows, and supported from the outside using hydraulic actuated piers anchored to ground. This systems acts as a further layer of isolation and is used to correct macroscopic positioning correction like the one needed to correct for the moon tidal forces.

%Four 1-m long silica fibers, two per side, connect the test mass to an optic of the same size and shape above it. The fibers are directly welded to both optics in a monolithic assembly to reduce dissipation and consequently thermal noise. The upper optic is then suspended by four steel wires that attach to the tip of as many \textit{blade springs}, flexible metal blades providing vertical isolation. The blade springs are anchored to a metal mass, which is in turn attached to the suspension structure by another four wires and four blade springs.

\paragraph*{Laser}
At freqeuncies above about ??? Hz, the interferometer sensitivity is limited by shot noise in the laser. The relative impact of the shot noise scales as the inverse of the sqare root of the power: increasing the laser power is thus a conceptually straightforward way of improving the sensitivity in the shot-noise limited band. The Advanced LIGO laser source is designed to deliver a maximum of 180\.W of laser power, as opposed to the 35\.W used in eLIGO. To obtain this, a the eLIGO laser unit is coupled to an high power laser amplifier (say more about this). The beam is per-stabilized in power, using a reference photodiode, and in frequency, using a thermally controlled refernece optical vacity, before being handed off to the Input Optic subsystem.

\paragraph*{Thermal compensation system}\label{sec:presfut:ground:2gen:aLIGO:TCS}
Despite the stringent requirement on the optical absorption of bulk and coating materials of the optics, the high power levels circulating in the interferometer result in a non-negligible amount of heat released into the optics. Due to the poor heat conduction in vacuum, this induces important thermal gradients that can modify the optical parameters of the system via two main effects: thermal lensing in the bulk material, due to the temperature dependence of the index of refraction, and surface distortion of the high-reflective surfaces, due to thermal expansion. This effect is especially important for the input test masses, since they transmit the most power, and their high-reflective surface is part of the arm cavities of the interferometer.

Initial LIGO already included a thermal compensation system based on CO2 laser that could strategically deliver power to compensate for thermal gradients. This relatively simple system, affected by poor performance and noisy operation, has been radically redesigned for Advanced LIGO. The new thermal compensation system is comprised of two type of actuators: ring heaters, one for each test mass, and CO2 lasers, one for each input test mass. The ring heaters are annular IR emitters that heat the test mass barrel. Their placement is tuned in such a way that the additional thermo-mechanical stress induces a convex deformation of the optic high-reflective surface, thus compensating the central bulge due to the heat released by the laser beam. At the same time, since the laser beam release heat along the axis of the optics, heating from outside reduces the overall thermal gradient and consequently the thermal lensing through the optic.

The ring theater are intended to be tuned to optimally compensate the test mass surface distortion. In general, this results in an only partial compensation fo the thermal lensing. To correct for any residual lensing, the compensation plate is heated by a CO2 laser projector whose pattern can be optimized through the usage of custom masks. The combination of ring heater and CO2 projector on each input thest mass is thus design to fully compensate the geometry of both the reflected and transmitted beams.

The end test masses only have ring heaters, since the transmitted light is used for diagnostic purposes and is not critical to the performance of the instrument.

The thermal compensation system also includes Hartman sensors to monitor the thermal state of the test masses. This kind of sensors works by blocking an incoming beam with a grid of apertures, which result in an array of points being projected on a CCD sensor. The position of each points on the CDD depends on the local curvature of the wavefront of the beam at the corresponding aperture. Any distortion of the wavefront with respect to a reference state is recorded as a relative displacement of the points on the CCD.
In Advanced LIGO, the Hartman sensors use green laser beams injected from the AR face of the test masses, reflected back by the HR face and then directed towards the mask. Each sensor thus monitors the integral effect of a double pass through the substrate thermal lens, and the reflection off the distorted HR surface.

\begin{itemize}
\item Sesmic isolation upgrade (Giac)
\item optical layout (Paul)
\item Laser power upgrade (Giac)
\item sensing and control (Paul)
\item Thermal compensation (Giac)
\item signal recycling (Paul)
\end{itemize}

The optical layout of Advanced LIGO is different from the initial LIGO layout in several ways. 
Perhaps the most fundamental change to the optical layout is the addition of a signal recycling 
mirror between the anti-symmetric side of the beam splitter and the optical detection port. 
In its current configuration this oft-called signal recycling mirror is actually tuned such as to 
increase the bandwidth of the detector. As such, a more apposite name might be the signal extraction mirror. 

The signal extraction mirror forms a new cavity within Advanced LIGO; the signal recycling cavity. 
Both the signal recycling cavity and the power recycling cavity in Advanced LIGO are designed to be 
geometrically \emph{stable}, by which it should be understood that the round-trip Gouy phase in the cavity 
is significant, and thus higher-order spatial modes are non-degenerate. 
This is contrast to the power recycling cavity in initial LIGO, which was only marginally stable. 
The advantages of the stable recycling cavity design have been clear during the commissioning of 
Advanced LIGO, where commissioning of the length and alignment sensing and control systems has 
been a much smoother process than in initial LIGO. 

Another important geometric change to the optical layout between initial LIGO and Advanced LIGO is 
in the arm cavities. There was a drive towards using larger beam spot sizes on the mirrors in Advanced 
LIGO in order to mitigate the effects of thermal noise. In general there are two cavity geometry solutions available 
that will give a specific beam spot size on the mirrors for a two-mirror cavity of fixed length. The initial LIGO 
arm cavities were designed with a large beam waist inside the cavities, resulting in a cavity g-factor of XXX. 
Advanced LIGO uses the alternative solution of having a small beam waist size in the cavities, giving a cavity g-factor of XXX. 
One of the advantages of the Advanced LIGO arm cavity design is that the relatively large divergence angle of the cavity 
eigenmode reduces the sensitivity of the interferometer to beam tilts (check). 

Several additional optical subsystems have been added in the upgrade from initial LIGO to Advanced LIGO. 
During the enhanced LIGO phase (shortly before initial LIGO went offline for the major upgrade to aLIGO) an output
mode cleaner was added at the output port. The output mode cleaner is a crucial component of the DC readout 
scheme which was first demonstrated in GEO600, and which was determined to be a more optimal solution for 
readout of the gravitational wave signal than the previously used RF heterodyne readout scheme. The output mode 
cleaner subsystem was retained in the aLIGO optical layout, and takes the form of a suspended fixed spacer cavity 
with a bow tie configuration. The output mode cleaner has the essential function of removing RF sidebands and higher-order 
spatial modes from the light incident on the photodiode, this mitigating their impact on the shot noise sensitivity. 

A great effort was made in the upgrade to Advanced LIGO to make the lock acquisition process more deterministic than 
stochastic. Part of this effort was the inclusion of the arm length stabilization (ALS) subsystem. This subsystem consists uses 
green frequency-doubled Nd:YAG beams which are phased locked to the main laser to independently control the arm cavities during 
lock acqusition of the central dual-recycled Michelson interferometer (DRMI). Once the DRMI is locked the ALS can be used 
methodically bring the arms to resonance, bringing the full interferometer to the ideal operating point.

interferometric sensing and control (Paul)

\begin{itemize}
\item length sensing and control, PDH
\item sideband frequencies, resonances in certain cavities
\item Schnupp asymmetry
\item DC offset
\item ESDs
\item ASC
\end{itemize}

signal recycling. I actually think this should maybe just go under optical layout.
Maybe also have a section for noise budget, or sensitivity curve?








Aside from the addition of the signal recycling mirror, the most notable change from the initial LIGO optical 
layout is the stable design of the recycling cavities.


\subsubsection{Advanced Virgo}
\begin{itemize}
\item Sesmic isolation upgrade?
\item Laser power upgrade
\item Optical layout
\item sensing and control
\item Thermal compensation
\item signal recyling
\end{itemize}