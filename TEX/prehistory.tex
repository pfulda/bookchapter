\subsection{Prehistory of ground-based interferometric gravitational wave detectors}\label{subsec:prehistory}

The very earliest history of interferometric gravitational wave detectors lies in the experiments 
of the American physicists Albert Michelson and Edward Morley in the summer of 1887. Although 
the goal of the famous Michelson-Morley experiment was not to detect gravitational waves (indeed 
the theory that predicts their existence was still several decades away from being conceived), the basic 
design of their apparatus can still be found in every ground-based interferometric gravitational wave detector. 
Michelson and Morley used their interferometer to attempt to measure variations in the speed of 
light with direction of propagation, and as a result to measure the velocity of the Earth with respect to the 
luminiferous aether: a feature of the prevailing physical theories of the time. By the time of their 1887 measurement, 
their apparatus was deemed capable of measuring shifts of about 1\% of a fringe; a remarkable feat given the 
technology available. 

The negative results of the Michelson-Morley experiment eventually paved the way for Einstein's theory of special 
relativity, in which the speed of light is invariant with propagation direction. It is somehow fitting that a variant 
of the same apparatus was used in 2016 to make the first direct measurement of gravitational waves, themselves 
a key prediction of Einstein's theory of general relativity. 

%There is also an interesting parallel... somehow link MM experiment trying to find relative velocity to ether 
%using speed of light, Einstein relativity saying there is no such ether, GR saying GWs travel at speed of light, 
%MI proving twice that there is no relative velocity for light!

Efforts to detect gravitational waves began in earnest with Joseph Weber's development of resonant bar detectors 
in the 1960s. This detection scheme relied on the excitation of resonant modes of a mass with a high mechanical 
quality factor by passing gravitational waves. The reliance on resonances of the test mass produced a detector 
with an extremely limited bandwidth, able only even in principle to detect the presence of a GW signal, and not 
to uncover detailed information about the nature of the sources of the the waves. 

Weber reported a series of detections throughout the 1960s. Efforts by Richard Garwin, Heinz Billing and others(?) 
to reproduce his results were fruitless, however, and by the 1970s the veracity of Weber's detection claims was widely doubted. 
These early claims of detection and their subsequent dismissal are likely to be at least partly responsible for the 
overtly scrupulous nature of the modern gravitational wave detection field, evidenced for example in the great lengths to which 
the LVC collaboration went before publishing their first detection paper.

In the 1960s the idea of using laser interferometers as gravitational wave detectors was developed more or less simultaneously 
in several places, by Joseph Weber himself, along with soviet physicists Mikhail Gertsenshtein and Vladislav Pustovoit. 
It was not until 1968, however, that Rainer Weiss first performed a detailed noise analysis of a laser interferometer in the context 
of gravitational wave detection, considering all of the fundamental noise sources that still limit detectors to this day. 
It was this study that really demonstrated the feasibility of using large-scale laser interferometers for gravitational wave detection, and 
it was instrumental in securing funding for the further development of the technology, prototype interferometers, and eventually LIGO itself. 

The late 1960s through the early 1990s was the era of prototype interferometric gravitational wave detectors, beginning with 
Robert Forward (a former graduate student of Weber), through Weiss' prototype at MIT, the Garching prototype developed by 
Heinz Billing, and a prototype in Glasgow lead by Ronald Drever and James Hough. 
It was clear from Weiss' initial 
study that although reaching the required sensitivity to detect gravitational waves was possible in principle, a huge technological effort would 
be required to make that potential a reality. 
Prototype detectors were an essential part of that technology development. Funding a full-scale 
observatory was still deemed too risky during this era, and in any case the technology simply was not at a mature enough stage to give them 
a reasonable chance of detecting gravitational waves. These prototypes also provided the function 
of training young scientists in the methods and concepts that would be instrumental in 
designing, building and operating gravitational wave detectors. 

At the beginning of the 1980s the two projects that would eventually join together to form LIGO were initiated: 
a design study for a kilometer scale interferometer at MIT, and a 40m 
prototype interferometer at Caltech. Work continued on these projects, and the LIGO project persistently applied 
for funding through the late 1980s, eventually obtaining its full funding to the tune of nearly \$400 million in 1994. 

Italian? French? Japan?
