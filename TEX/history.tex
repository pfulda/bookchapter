\newpage
%NOTE: we should clarify with P&P if they want this to be reviewed/approved, and at which stage
\subsection{History of ground-based interferometric gravitational wave detectors}\label{subsec:prehistory}
The history of gravitational wave detection is one of fiendishly bright ideas, perseverance in the face of incredible 
technical challenges, and the growth of a global community of scientists focused on a singular goal of detecting 
gravitational waves and using them to learn more about the universe. At the time of writing the first part of this 
charge was recently achieved, with the detection of a gravitational wave signal unmistakably generated from a binary 
black hole coalescence by the Advanced LIGO detectors on the 15th of September 2015~\cite{GW150914}. To 
some this achievement is the culmination of many years of hard work and dedication. To others, this is merely 
the beginning of a new era of gravitational wave astronomy. 

The very earliest history of interferometric gravitational wave detectors lies in the experiments 
of the American physicists Albert Michelson and Edward Morley in the summer of 1887. Although 
the goal of the famous Michelson-Morley experiment was not to detect gravitational waves (indeed 
the theory that predicts their existence was still several decades away from being conceived), the basic 
design of their apparatus can still be found in every ground-based interferometric gravitational wave detector. 
Michelson and Morley used their interferometer to attempt to measure variations in the speed of 
light with direction of propagation, and as a result to measure the velocity of the Earth with respect to the 
luminiferous aether: a feature of the prevailing physical theories of the time. By the time of their 1887 measurement, 
their apparatus was deemed capable of measuring shifts of about 1\% of a fringe; a remarkable feat given the 
technology available. 

The negative results of the Michelson-Morley experiment eventually paved the way for Einstein's theory of special 
relativity, in which the speed of light is invariant with propagation direction. It is somehow fitting that a variant 
of the same apparatus was used in 2016 to make the first direct measurement of gravitational waves, themselves 
a key prediction of Einstein's theory of general relativity. 

%There is also an interesting parallel... somehow link MM experiment trying to find relative velocity to ether 
%using speed of light, Einstein relativity saying there is no such ether, GR saying GWs travel at speed of light, 
%MI proving twice that there is no relative velocity for light!

Efforts to detect gravitational waves began in earnest with Joseph Weber's development of resonant bar detectors 
in the 1960s. This detection scheme relied on the excitation of resonant modes of a mass with a high mechanical 
quality factor by passing gravitational waves. The reliance on resonances of the test mass produced a detector 
with an extremely limited bandwidth, able only even in principle to detect the presence of a GW signal, and not 
to uncover detailed information about the nature of the sources of the waves. 

Weber reported a series of detections throughout the 1960s. Efforts by Richard Garwin, Heinz Billing and others 
to reproduce his results were fruitless, however, and by the 1970s the veracity of Weber's detection claims was widely doubted. 
These early claims of detection and their subsequent dismissal are likely to be at least partly responsible for the 
overtly scrupulous nature of the modern gravitational wave detection field, evidenced for example in the great lengths to which 
the LVC collaboration went in order to verify their results before publishing their first detection paper~\cite{GW150914}.

In the 1960s the idea of using laser interferometers as gravitational wave detectors was developed more or less simultaneously 
in several places, by Joseph Weber himself, along with soviet physicists Mikhail Gertsenshtein and Vladislav Pustovoit. 
It was not until 1972, however, that Rainer Weiss first performed a detailed noise analysis of a laser interferometer in the context 
of gravitational wave detection, considering all of the fundamental noise sources that still limit detectors to this day~\cite{Weiss1972}. 
It was this study that really demonstrated the feasibility of using large-scale laser interferometers for gravitational wave detection, and 
it was instrumental in securing funding for the further development of the technology, prototype interferometers, and eventually LIGO itself. 
%FIX: here LIGO is mentioned but never introduced. Probably the concern will go away once we write the introductory part.
The late 1960s through the early 1990s was the era of prototype interferometric gravitational wave detectors, beginning with 
Robert Forward (a former graduate student of Weber), through Weiss' prototype at MIT~\cite{MITprototype}, the Garching prototype developed by 
Heinz Billing~\cite{Shoemaker1988}, and a prototype in Glasgow lead by Ronald Drever and James Hough~\cite{JIF}. 
It was clear from Weiss' initial study that although reaching the required sensitivity to detect gravitational waves 
was possible in principle, a huge technological effort would 
be required to make that potential a reality. 
Prototype detectors were an essential part of that technology development. Funding a full-scale 
observatory was still deemed too risky during this era, and in any case the technology simply was not at a mature enough stage to give them 
a reasonable chance of detecting gravitational waves. These prototypes also provided the function 
of training young scientists in the methods and concepts that would be instrumental in 
designing, building and operating gravitational wave detectors. 

At the beginning of the 1980s the two projects that would eventually join together to form LIGO were initiated: 
a design study for a kilometer scale interferometer at MIT, and a 40m 
prototype interferometer at Caltech~\cite{caltech40m}. Work continued on these projects, and the LIGO project persistently applied for funding through the late 1980s.

%ISSUE: should we cite the LIGO paper in Science 1992? Where?
In 1992, NSF approved funding for the construction of the two experimental facilities of the LIGO (Laser Interferometer Gravitational-wave Observatory) project\cite{Abramovici_1992}.
After a short period, however, it became clear that the team did not have the expertise and organizational skills to manage a project of this size, and funding was frozen following a review from a NSF oversight panel.
The situation was solved in winter 1994, when Barry Barish was appointed director of the project. Drawing on his previous experience with large scale scientific projects, Barish and his team put together a comprehensive and convincing management plan.
The revised plan, that among other things increased the cost estimate from 250 to more than 290 million dollars, was approved again in 1994, despite skepticism and some strong opposition from part of the physics and astronomy communities. 
Many thought that the investment, the largest ever made by NSF on a  single project, was too risky, that it would needlessly drain resources from other research (which in fact did not happen, mostly thanks to the able political and financial planning of the NSF director Eric Bloch), and that the chances of success would be almost non existent.
History would eventually prove them wrong.

Ground was immediately broken in Hanford, WA, and the following year in Livingston, LA; 
the construction of the buildings and of the vacuum system, by some measures the biggest 
ever built at the time, took almost five years. Separated by more than 3000\,km, the two 
experimental sites shared the same basic design: a 4\,km long L shape structure; they 
however differed in orientation (both have one arm aligned with the great circle joining the two sites, but the other arms are anti-parallel), and for the fact that the one in Hanford was 
designed to accommodate two parallel interferometers, 2\,km and 4\,km long respectively, in the same vacuum system. 
The installation of the scientific equipment started in 1999 and was completed by the end of 2000. 
In the meantime, a broader scientific community had grown around the LIGO project, 
and had taken the shape of two institutions: the LIGO laboratory, in charge of managing 
the facility and most of the research and development directly aimed at improving the 
instruments, and the LIGO Scientific Collaboration (LSC), formed by research groups 
around the world involved in technical and scientific research related to LIGO.

While construction of the two LIGO detectors was ongoing in the US, parallel efforts 
were being pursued in Europe. A French-Italian collaboration secured funding from CNRS and INFN for a 
similar facility to be built in Cascina, near Pisa, Italy.
The construction of the 3-km long Virgo interferometer started in 1996 and was completed in 2003.
During this period, the European Gravitational Observatory (EGO) consortium was created to operate the detector 
and promote gravitational research in Europe.

The UK and Germany also joined forces to build the a large scale interferometer; the full-size project was not funded, but was de-scoped to a slightly smaller version named GEO600 (due to its 600 meters long arms), 
whose construction near Hannover, Germany, started in 1995. The members of the GEO600 project were also founding members of the LSC.

Smaller scale interferometers, mainly intended as prototypes, were built or 
proposed in other parts of the world. In particular, ACIGO in Australia and 
CLIO and TAMA300 in Japan. CLIO was the first detector to implement cryogenic operation and demonstrate a reduction in thermal noise, although its limited size did not allow it to reach a competitive sensitivity for gravitational wave detection~\cite{CLIOthermal}.

%TODO: add a paragraph stating that the different scientific communities realized that collaboration was essential to succeed, and so grew closer and closer. So much than now LIGO and VIRGO are basically a single scientific collaboration, although the management of the projects remain separated.