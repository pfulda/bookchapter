\newpage
\subsection{The first generation of interferometric detectors}\label{subsec:history}

In 1994, NSF approved funding for the construction of the two experimental facilities of the LIGO (Laser Interferometric Gravitational-wave Observatory) project, despite skepticism and some strong opposition from part of the physics and astronomy communities: many thought that the investment, the largest ever made by NSF on a single project, was too risky, that it would needlessly drain resources from other research, and that the chances of success would be almost non existent. History will prove them wrong.
Ground was broken the same year in Hanford, WA, and the following one in Livingston, LA; the construction of the buildings and of the vacuum system, by some measures the biggest ever built at the time, took almost five years. Separated by almost 3000 km, the two experimental sites shared the same basic design, and 4-km long L shape structure; they however differed for the orientation, and for the fact that the one in Hanford was design to accommodate two parallel interferometers in the same vacuum system. The installation of the scientific equipment started in 1999 and was completed by 2002. In the meanwhile, a broader scientific community had grown around the LIGO project, and had taken the shape of two institutions: the LIGO laboratory, in charge of managing the facility and most of the research and development directly aimed at improving the instruments, and the LIGO Scientific Collaboration (LSC), formed by research groups around the world involved in technical and scientific research related to LIGO.
While construction of the two LIGO detectors was ongoing in the US, parallel efforts were being pursued in Europe. A French-Italian collaboration secured funding for a similar facility to be built in Cascina, near Pisa, in Italy. The construction of the 3-km long VIRGO interferometer started in 1996 and was completed in 2003. During this period, the European Gravitational Observatory (EGO) consortium was created to operate the detector and promote gravitational research in Europe. United Kingdom and Germany also joined forces to build the a large scale interferometer; the full-size project was not funded, but was de-scoped to a slightly smaller version named GEO600 (due to its 600 meters long arms), whose construction near Hannover, Germany, started in 1995.
Smaller scale interferometers, mainly intended as prototypes, were build or proposed in other parts of the world. In particular, ACIGO in Australia and CLIO and TAMA300 in Japan.
Among the major players, the first large-scale interferometers to start science observations were the LIGO detectors in 2002, alternating 

%TODO: things that can (must) be added, in loose order of importance
% names of major personalities
% name of institutions
% references
% detail of single and joint science runs of major detectors
% collaboration between LSC and VIRGO
