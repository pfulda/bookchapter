\subsection{Design and operation of the first generation of interferometric gravitational wave detectors}\label{subsec:1stgen}

Despite the extensive research and development effort put forward, and the experience acquired over several years on smaller prototypes, it was clear to the gravitational wave community, and to the funding agencies, that building and operation of km-scale interferometers at their design sensitivity was a high risk, although potentially high gain, endeavor. While the infrastructures, by far the largest voice in the initial budgets, were built to be able to support future upgraded detectors, the first generation of interferometers were designed to be relatively simple, reducing the odds of an actual detection but increasing the chances of their successful commissioning and operation. In particular, while the adopted topology was very similar to the one used in the subsequent generation, the complexity of many subsystems were kept at a minimum.

The initial LIGO detectors were power-recycled, Fabry-Perot interferometer. An out-of-vacuum, 10\,W pre-stabilized laser was phase modulated to add three sidebands for alignment and sensing control. It was then spatially filtered and further stabilized in power and frequency by an in-vacuum input-optics section: this included a 12\,m round-trip suspended triangular optical cavity, referred to as input mode cleaner, a low loss Faraday isolator, and a suspended telescope to match the beam mode to that of the rest of the interferometer. The beam was then injected into the recycling cavity (ref to appropriate section), which increased the input power seen by the rest of the interferometer by a factor of about 50; finally, the power reached about 20 kW in the impedance matched arm cavities, designed to have a finesse of 220. There was no signal recycling cavity, and the strain signal was obtained using RF readout of the interferometer output.

The vacuum system layout, designed to also accommodate subsequent upgrades of the detectors, was based on a series of vacuum chambers of either of two types: horizontal ones, for the input and output optics, and vertical ones for the core optics. Both type are cylinders about 2\,m in diameter and 3\,m in length or height, with large access ports for easy installation of heavy equipment. The corner station at Livingston hosts 6 horizontal chambers, 3 on the input and 3  on the output branch, and 3 vertical chambers, with another two hosted in the end stations. All chambers, including the ones at the end stations, are connected by 1.2\,m diameter vacuum tubes. At Handord, the need to accommodate a second interferometer required to double the number of chambers, although most of the vacuum tubes were shared by the two laser beams and only minimal additions were needed to connect the extra chambers.
%NOTE: where the table in the TM chambers the same?
Each chamber was equipped with an optical table, passively isolated from seismic vibrations by a 4-stage spring-mass system, providing about 8 orders of magnitude isolation at 100 Hz. Single pendulum suspension were used to support the most critical optical components, and in particular the 10\,Kg, 25\,cm diameter end mirrors of the Fabry-Perot arm cavities, referred to as the input and output test masses; for these optics, the pendulum suspensions provided further 4 orders of magnitude suppression of ground motion at 100 Hz. In Livingston, where the ground motion was sensibly higher than in Hanford, an out-of-vacuum pre-isolation system was added to contribute another factor 10 suppression between 0.1 and 10\,Hz.

%NOTE: is the following paragraph true? Dig out a virgo design paper...
The VIRGO detector, in Italy, was based on an essentially equivalent design, with the major difference being the adoption of 7-stages, 6\,m high three dimensional suspensions, named \textit{super-attenuators}, to isolate the input and output test masses. The \textit{super-attenuators} were designed to provide at least 10 orders of magnitude isolation from ground motion down to 4 Hz.

GEO600, on the other hand, adopted quite different design choices, and pioneered a number of innovative technologies most of which would later be integrated in the larger detectors: instead of faby-perrot cavity, it employed folded arms, a topology in which the end of the arms are occupied by folding mirrors that send the laser back towards the end test masses located close to the beam splitter; it was the first one to couple a signal recycling cavity to leverage resonant signal extraction; it also employed DC readout, rather than the more conventional RF homo-dyne readout; finally, it was the first one to employ squeezing, a technique used to shape quantum fluctuations and obtain a reduction in relative shot noise equivalent to that of a higher power laser.

%\begin{figure}[htb]
%\includegraphics[scale=0.4\textwidth]{}
%\caption{Schematics of GEO, LIGO, Virgo detectors here.}
%\label{fig:opticallayouts}
%\end{figure}
\missingfigure[figwidth=12cm]{Schematics of GEO, LIGO, Virgo detectors to go here.}

TAMA300, in Japan, ...

In 2002, the LIGO detectors and GEO600 reached a sensitivity adequate for collecting science data, although still far from the design goal. 
The first science run took place between August and September 2002 with a conventional range, intended as the maximum distance at which a standard NS-NS coalescence could be detected with a signal-to-noise ratio equal to 8, of 100 kpc, more than two orders of magnitude less than the design value of 18 Mpc. In the years that followed, LIGO conducted other 4 science runs, interrupted by commissioning periods that steadily and consistently improved the performance until it finally reached full design sensitivity in 2006. More on eLIGO and shutdown in 2010.



% TODO:
% include list of runs (including partners)
% Include eLIGO
% More on GEO600 as a pioneer



%and the vast majority of the initial budget. This was indeed a conscious choice: while the initial scientific equipment was designed to be a relatively simple version, charged with the main task of proving that such a large scale interferometer could be built and operated successfully at design sensitivity, but with slim chances of actually detecting a gravitational wave, the infrastructure was expected to support future upgrades of the detector until it could eventually live up to the "Observatory" definition used in its acronym.