\subsection{Design and operation of the first generation of interferometric gravitational wave detectors}\label{subsec:1stgen}

Despite the extensive research and development effort put forward, and the experience acquired over several years on smaller prototypes, it was clear to the gravitational wave community, and to the funding agencies, that building and operation of km-scale interferometers at their design sensitivity was a high risk, although potentially high gain, endeavor. While the infrastructures, by far the largest voice in the initial budgets, were built to be able to support future upgraded detectors, the first generation of interferometers were designed to be relatively simple, reducing the odds of an actual detection but increasing the chances of their successful commissioning and operation. In particular, while the adopted topology was very similar to the one used in the subsequent generation, the complexity of many subsystems were kept at a minimum.

The initial LIGO detectors were power-recycled, Fabry-Perot interferometer. An out-of-vacuum, 10\,W pre-stabilized laser was phase modulated to add three sidebands for alignment and sensing control. It was then spatially filtered and further stabilized in power and frequency by an in-vacuum input-optics section: this included a suspended optical cavity, referred to as input mode cleaner, a low loss Faraday isolator and a mode-matching telescope formed by suspended mirrors. The beam was then injected in the recycling cavity (ref to appropriate section), with a finesse of ... ; the power reached about ... kW in the impedance matched arm cavities, designed to have a finesse of about ... . There was no signal recycling cavity, and the strain signal was obtained using RF readout of the interferometer output.
The vacuum system was designed to accommodate the equipment in a series of vacuum chambers, each about 3 meters in size. There were 3 chambers along the input path, between the laser and the beams-splitter; one chamber for the beam splitter; one each for the two input and the two output test masses; and two chambers along the output path. The chambers were connected by vacuum tubes about 1 m in diameter. In Hanford, the vacuum layout was modified by the addition of a duplicated sections for the second interferometer.
Each chamber was equipped with an optical table passively isolated from seismic vibrations by a 3-stage spring-mass system, providing about ??? orders of magnitude isolation at ??? Hz. Single pendulum suspension were used to support the most critical optical components, and in particular the 20-Kg, 10" diameter end mirrors of the Fabry-Perot arm cavities, referred to as the input and output test masses.

The VIRGO detector was based on an essentially equivalent design, with the major difference being the adoption of 7-stages pendula, named \textit{super-attenuators}, to isolate the input and output test masses. The \textit{super-attenuators} were designed to provide at least 10 orders of magnitude isolation from ground motion down to 4 Hz.

GEO600, on the other hand, adopted quite different design choices, and pioneered a number of innovative technologies most of which would later be integrated in the larger detectors: instead of faby-perrot cavity, it employed folded arms, a topology in which the end of the arms are occupied by folding mirrors that send tha laser back towards the end test masses located close to the beam splitter; it was the first one to couple a signal recycling cavity to leverage resonant signal extraction; it also employed DC readout, rather than the more conventional RF homo-dyne readout; finally, it was the first one to employ squeezing, a technique used to shape quantum fluctuations and obtain a reduction in relative shot noise equivalent to that of a higher power laser.

TAMA300, in Japan, ...

In 2002, the LIGO detectors and GEO600 reached a sensitivity adequate for collecting science data, although still far from the design goal. 
The first science run took place between August and September 2002 with a conventional range, intended as the maximum distance at which a standard NS-NS coalescence could be detected with a signal-to-noise ratio equal to 8, of 100 kpc, more than two orders of magnitude less than the design value of 18 Mpc. In the years that followed, LIGO conducted other 4 science runs, interrupted by commissioning periods that steadily and consistently improved the performance until it finally reached full design sensitivity in 2006. More on eLIGO and shutdown in 2010.

Although no gravitational wave signal had been detected, important scientific outcome had been produced, mostly in the form of upper limits to the amplitude of GW radiation investing the Earth; much more importantly, the results of the initial detectors had proven the ability of the scientific community to build and operate such large scale interferometers at their design sensitivity. Time were ripe to move on with the construction of the advanced detectors, designed to have sensitivities sufficient to see at least a few events per year even according to the most pessimistic rate estimates.

The LIGO and VIRGO projects both followed a similar approach: a complete shutdown of the facilities and replacement of all subsystems with more advanced versions. More powerful laser, better seismic isolation systems, bigger optics, the introduction of a signal recycling cavity and a more sensitive readout system.
In Japan, the TAMA300 project evolved in LCGT, later renamed KAGRA: a 3-km scale detector to be build in the Kamioka mines near...,and designed to be operated a cryogenic temperature.
GEO600, despite its limited sensitivity at low frequency, opted to remain the only active detector while the other projects were undergoig construction and upgrades. In the years between 2010 and 2015 it alternated observing runs and commissioning periods, continuing his role of pioneering facility for technologies suitable to be later integrated in the larger interferometers.


% TODO:
% include list of runs (including partners)
% Include eLIGO
% More on GEO600 as a pioneer

 

%and the vast majority of the initial budget. This was indeed a conscious choice: while the initial scientific equipment was designed to be a relatively simple version, charged with the main task of proving that such a large scale interferometer could be built and operated successfully at design sensitivity, but with slim chances of actually detecting a gravitational wave, the infrastructure was expected to support future upgrades of the detector until it could eventually live up to the "Observatory" definition used in its acronym.