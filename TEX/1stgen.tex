\subsection{Design and operation of the first generation of interferometric gravitational wave detectors}\label{subsec:1stgen}

Despite the extensive research and development effort put forward, and the experience acquired over several years on smaller prototypes, it was clear to the gravitational wave community, and to the funding agencies, that building and operation of km-scale interferometers at their design sensitivity was a high risk, although potentially high gain, endeavor. While the infrastructures, by far the largest voice in the initial budgets, were built to be able to support future upgraded detectors, the first generation of interferometers were designed to be relatively simple, reducing the odds of an actual detection but increasing the chances of their successful commissioning and operation. In particular, while adopted the topology was very similar to the one used in the subsequent generation, the complexity of many subsystems were kept at a minimum.
The initial LIGO detector were power-recycled, Faby-perrot interferometer. A 10 W pre-stabilized laser was shaped and further stabilized in power and frequency by an input-optics section including a suspended mode-cleaner and a low loss, in vacuum faraday isolator. The beam was then injected in the power recycling cavity with a finesse of ... . 


and charged with the main task of proving that such a large scale interferometer could be built and operated successfully at design sensitivity, but with slim chances of actually detecting a gravitational wave

choice was made to keep the complexity of the instrument relatively simple, to ease the commissioning and operation tasks at the cost of 
the design of the first generation of large-scale interferometers was left relatively simple to limit the complexity of the instrument and ease the commissioning process, at the cost of a reduced sensitivity.

 

%and the vast majority of the initial budget. This was indeed a conscious choice: while the initial scientific equipment was designed to be a relatively simple version, charged with the main task of proving that such a large scale interferometer could be built and operated successfully at design sensitivity, but with slim chances of actually detecting a gravitational wave, the infrastructure was expected to support future upgrades of the detector until it could eventually live up to the "Observatory" definition used in its acronym.