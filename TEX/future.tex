\subsection{The future of ground based interferometric gravitational wave detection}\label{subsec:future}

After the announcement of the first detection of gravitational waves by the LIGO collaboration, the anticipation for an extended network of detectors to come online became even more pressing. Now that the possibility of detecting gravitational waves had been proven, the scientific community was looking forward at the science that could be made by collecting and analyzing a large number of events with well constrained parameters. A network of three or more detectors would improve on a number of key factors, among which better duty cycles, better parameter estimation and much ore precise sky localization.

Virgo was already expected to join the second LIGO observing run starting in summer 2016, and despite some installation and commissioning issues, the LIGO and Virgo communities are working hard to make it happen.

Kagra wasn't originally planned to join the network with meaningful sensitivity before 2018, and various delays were pushing the schedule back even more. The managing team went trough a redefinition of the schedule in winter 2016, and outlined a plan to skip an intermediate commissioning phase and accelerate the path to the final, cryogenic version of the detector. In spring 2016 the interferometer was locked for the first time in a simple Michelson configuration with no arm cavity or recycling cavities, but using the full 3\,km arm-length.

The announcement of GW150914 also gave a decisive impulse to the LIGO India project. As for initial LIGO, the Advanced LIGO project included two interferometers to be installed in the Hanford site vacuum system; unlike for Initial LIGO, the second interferometer was designed to be 4\,km long, rather than 2\,km. Shortly after the installation phase began, an idea started to take hold in the community: what if, instead of building two co-located interferometers that would be affected by the same disturbances and not add much to the science, the third instrument could be moved to a completely different location? If a country was willing to invest in the construction of the infrastructure and vacuum system, it would be rewarded with the ability to jump to the forefront of gravitational wave detection by having the Advanced LIGO project transfer to them all the instrumentation that was designed and built for the second Hanford detector. Besides likely improving the overall network duty cycle, a strategic placement of the additional detector would also great improve the parameter estimation and localization capability of the network~\cite{something}.
Australia was initially identified as a possible partner; when funding constraints made it clear that the deal could not happen on fruitful time scales, India stepped in. The project seemed to move quickly at the beginning, with the INDIGO (the Indian gravitational wave community) managements putting forward a great effort to train their scientist, complete site surveys and making all other necessary preparation, with great help and support by the LIGO management; unfortunately, a change in government slowed process almost to a halt in 2015. If was only after the announcement of the first discovery by the LIGO project that the Indian government approved the project. Although the detector is not projected to come online before 2022, this represents an important success that will strategically expand the network of GW detectors and the list of countries involved.

In the meanwhile, the LIGO project is already researching possible upgrades, to be developed and installed in a few years time-frame in the current detectors without the need of a complete rebuild. Such upgrades includes frequency dependent squeezing, heavier test masses, improved coating, suspensions with reduced thermal noise and strategies to subtract Newtonian noise. Proposals are also being considered to cool down the test masses, thus reducing thermal noise in coating and suspensions.

Despite the factor of few improvement in overall sensitivity attainable with the above mentioned upgrades, the GW community is start to realize that a substantial gains in maximum and low-frequency sensitivity will require to abandon the current infrastructures. Different studies are being carried out to shape the concept of the next generation of ground detector, including a proposal in the LIGO community to move to a detector of essentially the same design, but substantially increased length. The most advanced and well studied concept, however, has been developed by the European Community for a detector dubbed Einstein Telescope.
The observatory would be build about 150\,m underground, in a system of galleries forming a 10-km long horizontal equilateral triangle. Each vertex of the triangle would host the corner station of a detector, and the end test masses of the other two. Each detector would actually be comprised of two dual-recycled, resonant arm-cavities interferometers, sharing the same geometrical arrangement but optimized for the different frequencies bands: a cryogenic, low power one for the lower frequencies, and a room temperature more powerful version for the higher ones.

Compared to any one of the current observatories, the six interferometers combined would exhibit a much more uniform antenna pattern, in respect to both sky-position and polarization of the sources, improve the maximum sensitivity by more than a factor of 10 and allow to observe signals down to about 1\,Hz. Their proposed design is based on technologies that are currently state of the art, or have a mature enough state of development to make solid prediction about their future performance possible; however, the ET community has made clear that one of the main goal of the proposal is that of building an infrastructure for an observatory expected to remain current for decades, while the hosted instrumentation is upgraded according to the latest developments.

\begin{itemize}
	\item Near/mid future
	\begin{itemize}
		\item LIGO India
		\item In-situ upgrades to LIGO
		\item Kagra
		\item Virgo? Not sure they have anything in mind before ET...
	\end{itemize}
	\item Long term
	\begin{itemize}
		\item LUNGO (or similar)~\cite{Dwyer2015}
		\item ET~\cite{Punturo2010}
	\end{itemize}
\end{itemize}
