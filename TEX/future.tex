\section{The future of ground based interferometric gravitational wave detection}\label{subsec:future}

After the announcement of the first detection of gravitational waves by the LIGO-Virgo collaboration, the anticipation for an extended network of detectors to come online has become even more marked.
Now that the possibility of detecting gravitational waves has been proven, the scientific community is looking forward to the science that can be done by collecting and analyzing a large number of events with well constrained parameters.
A network of three or more detectors will improve on a number of key factors, including coincident duty cycle, source parameters estimation and sky localization. The latter is fundamental to increase the odds of simultaneously observing an event in both the gravitational wave and EM spectra, thus enabling so called \textit{multi-messenger astronomy}.

Advanced Virgo was already expected to join the second LIGO observing run starting in fall 2016, and despite some installation and commissioning issues, the LIGO and Virgo communities are working hard to make it happen.

Japan is currently building a km-scale cryogenic detector called KAGRA in the Kamioka mine\cite{Ar2013_PRD_Aso}.
Although its design sensitivity is comparable to that of the advanced interferometers, it represents a bridge towards the next generation detectors thanks to its underground location and the operation at cryogenic temperature.
It was not originally planned for KAGRA to join the network with meaningful sensitivity before 2018, and various delays are pushing the schedule back even more. 
The managing team went trough a redefinition of the schedule at the beginning of 2016, and outlined a plan to skip an intermediate commissioning phase and accelerate the path to the final, cryogenic version of the detector. 
In spring 2016 the interferometer was locked for the first time in a simple Michelson configuration with no arm cavities or recycling cavities, but using the full 3\,km arm-length.

The announcement of the detection of GW150914 also gave a decisive impulse to the LIGO India project.
As for Initial LIGO, the Advanced LIGO project included two interferometers to be installed in the Hanford site vacuum system.
Unlike Initial LIGO, the second interferometer was designed to be 4\,km long, rather than 2\,km. 
Shortly after the installation phase began, an idea started to take hold in the community: what if, instead of building two co-located interferometers that would be affected by the same disturbances and not add much to the science output, the third instrument could be moved to a completely different location? 
If a country was willing to invest in the construction of the infrastructure and vacuum system, it would be rewarded with the ability to jump to the forefront of gravitational wave science by borrowing from the Advanced LIGO laboratory all the instrumentation that was designed and built for the second Hanford detector. 
Besides improving the overall network duty cycle, a strategic placement of the additional detector would also greatly enhance the parameter estimation and localization capability of the network~\cite{Klimenko_2016}.
Australia was initially identified as a possible partner, but when funding constraints made it clear that the deal could not happen on fruitful time scales, India stepped in.
The project moved quickly at the beginning, and the INDIGO (the Indian gravitational wave community) management put forward a great effort to train their scientists, complete site surveys and make all other necessary preparations, with substantial help and support from the LIGO management.
Unfortunately, a change in government slowed the process almost to a halt in 2015. 
It was only after the announcement of the first detection by the LIGO Laboratory that the Indian government approved the project. 
Although the detector is not projected to come online before 2022, this represents an important success that will strategically expand the network of large-scale ground-based gravitational wave detectors and the list of countries involved.

In the meanwhile, the LIGO Laboratory is already researching possible upgrades, to be developed and installed in a few years time-frame in the current detectors without the need of a complete rebuild.
Such upgrades include frequency dependent squeezing, improved coating, suspensions with reduced thermal noise and strategies to subtract Newtonian noise.
Proposals are also being considered to use heavier test masses, and possibly cool them down to reduce thermal noise in coating and suspensions.
%Plans are also underway to begin a design study for a next generation LIGO facility, tentatively titled \emph{LIGO cosmic explorer}, that may include longer beam tubes capable of housing a detector with a sensitivity 10 times that of Advanced LIGO across the detection band~\cite{Dwyer2015}. 

Despite the factor of few improvement in overall sensitivity attainable with the above mentioned upgrades, it is clear to the gravitational wave research community that a substantial gain in maximum and low-frequency sensitivity will require the development of new infrastructures. 
Different studies are being carried out to shape the concept of the next generation of ground based detectors, including a proposal in the LIGO community to move to a detector of essentially the same design, but substantially increased length~\cite{Dwyer2015}.

Research groups in Brasil, Argentina and Mexico are working on a proposal for the construction of an underground cryogenic detector, currently referred to as SAI (South American Interferometer);  its location in the southern hemisphere will substantially improve the antenna pattern and localization capabilities of the network.

The most advanced and well studied concept, however, has been developed by the European community for a detector dubbed Einstein Telescope~\cite{Punturo2010}.
The observatory would be built about 150\,m underground, in a system of galleries forming a horizontal equilateral triangle with 10\,km long sides. Three detectors would be co-located in the facility, with each vertex of the triangle hosting the corner station of one, and one end station of each of the other two.
Each detector would actually be comprised of two dual-recycled, Fabry-Perot Michelson  interferometers, sharing the same geometrical arrangement but optimized for two different frequency bands: a cryogenic, low power one for the lower frequency sensitivity, and a room temperature more powerful version for higher frequency sensitivity.

Compared to any one of the current observatories, the six interferometers combined would exhibit a much more uniform antenna pattern, in terms of both sky-position and polarization of the sources, improve the maximum sensitivity by more than a factor of 10 and allow to observe signals down to about 1\,Hz. 
Their proposed design is based on technologies that are currently state of the art, or have a mature enough state of development to make solid prediction about their future performance possible; however, the ET community has made clear that one of the main goal of the proposal is that of building an infrastructure for an observatory expected to remain current for decades, while the hosted instrumentation is upgraded according to the latest developments.

The nature of the field of gravitational wave detection is such that it may be many years before a new breakthrough technology with the potential to improve the strain sensitivity makes it from the conceptual stage to real implementation in a full scale detector. 
As such, research groups around the world are already working hard on developing the technologies of the future, perhaps a decade or more before they might make it into the vacuum enclosures and clean rooms of the future instruments. 
This constant drive for better sensitivity makes the field a very creative one, and one in which there can be expected to be many years of innovation and technical achievements still to come, even after the first direct detection of gravitational waves.