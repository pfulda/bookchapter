\subsection{The future of ground based interferometric gravitational wave detection}\label{subsec:future}

After the announcement of the first detection of gravitational waves by the LIGO detector, the anticipation for an extended network of detectors to come online become even more pressing. Now that the possibility of detecting gravitational wave had been proven, the scientific community was looking forward at the science that could be made by detecting a large number of events with well constrained parameters. A network of three or more detectors would improve on a number of key factors, among which better duty cycles, better parameter estimation and much ore precise sky localization.
Virgo was already expected to join the second LIGO observing run starting in summer 2016, and despite some installation an commissioning issue, the LIGO and Virgo communities were working hard to make it happen.
Kagra wasn't originally planned to join the network with meaningful sensitivity before 2018, and various delays were pushing the schedule back even more. The managing team went trough a redefinition of the schedule in winter 2016, and outlined a plan to skip an intermediate commissioning phase and accelerate the path to the final, cryogenic version of the detector.


\begin{itemize}
	\item Near/mid future
	\begin{itemize}
		\item LIGO India
		\item In-situ upgrades to LIGO
		\item Kagra
		\item Virgo? Not sure they have anything in mind before ET...
	\end{itemize}
	\item Long term
	\begin{itemize}
		\item LUNGO (or similar)
		\item ET
	\end{itemize}
\end{itemize}
