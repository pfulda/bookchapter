%%%%%%%%%%%%%%%%%%%%%%%%%%%%%%%%%%%%%%%%%%%%%%%%%%%%%%%%%%%%%%%%%%%%%%%%%%
%% For support : Yolande Koh, <ykoh@wspc.com.sg>                        %%
%% Review Volume (last updated on 20-4-2015)                            %%
%% Trim Size: 9in x 6in                                                 %%
%% Text Area: 7.35in (include runningheads) x 4.5in                     %%
%% Main Text: 10 on 13pt                                                %%
%% Technical Support: D. Rajesh Babu, <rajesh@wspc.com.sg>              %%
%%%%%%%%%%%%%%%%%%%%%%%%%%%%%%%%%%%%%%%%%%%%%%%%%%%%%%%%

%% My packages
\documentclass[master]{ws-rv9x6} % 'master' to combine all chapters

%\usepackage{subfigure}   % required only when side-by-side / subfigures are used
\usepackage{ws-rv-thm}   % comment this line when `amsthm / theorem / ntheorem` package is used
\usepackage[square]{ws-rv-van}   % numbered citation & references (default)
\usepackage{ws-index}     % required only when multiple indexes are used
\usepackage{booktabs}	 % professional-looking tables
\usepackage{bm}
\usepackage{color}
%\usepackage[square]{ws-rv-van}   % numbered citation & references (default)
\usepackage[colorlinks]{hyperref}
\usepackage[final ]{pdfpages}  % required for include PDF
\newindex{aindx}{adx}{and}{Author Index}       % author index
%\renewindex{default}{idx}{ind}{Subject Index}  % subject index
%\renewindex{idx}{ind}{Subject Index}  % subject index

%\usepackage[pdfpagelabels=false]{hyperref}  % \label, \ref and \cite are recommended

\usepackage{import} % Define import

\makeindex % What does this do ?

\definecolor{darkblue}{rgb}{0,0,0.3}
\hypersetup{citecolor=darkblue}
\hypersetup{linkcolor=darkblue}


%%%%%%%%%%%%%%%%%%%%%%%%%
% Macro definitions here
%%%%%%%%%%%%%%%%%%%%%%%%%
\newcommand{\be}{\begin{equation}}
\newcommand{\ee}{\end{equation}}
\newcommand{\beq}{\begin{equation}}
\newcommand{\eeq}{\end{equation}}
\renewcommand{\c}{\,,}
\newcommand{\p}{\,.}
\def\varM{\mathcal{M}}
\def\calO{\mathcal{O}}
\def\ud{\mathrm{d}}
\def\ui{\mathrm{i}}
\def\n{\nabla\!}
\newcommand{\red}{\textcolor{red}}
\newcommand{\jn}[1]{\textbf{\textcolor{green}{#1}}}
%%%%%%%%%%%%%%%%%%%%%%%%%

% --------------- user defined commands --------------

\newcommand{\bra}[1]{\langle #1|}
\newcommand{\ket}[1]{|#1\rangle}
\newcommand{\ddt}[1]{\frac{\partial #1}{\partial t}}
\newcommand{\eff}{_{\text{eff}}}
\newcommand\stxt[1]{_{\text{#1}}} % subscript
\newcommand\STXT[1]{^{\text{#1}}} % superscript
\newcommand{\ms}{\ \text{ms}}
\newcommand{\m}{\ \text{m}}
\newcommand{\nK}{\ \text{nK}}
\newcommand{\Hz}{\ \text{Hz}}
\newcommand{\kHz}{\ \text{kHz}}
\newcommand{\micron}{\ \mu\text{m}}
\newcommand{\rad}{\ \text{rad}}
\newcommand{\sqrtHz}{/\sqrt{\text{Hz}}}

\newcommand{\lambdabar}{\mbox{\makebox[-0.5ex][l]{$\lambda$} \raisebox{0.7ex}[0pt][0pt]{--}}}
\newcommand{\aap}{{Astron. Astrophys.}}
\newcommand{\apj}{{Astrophys. J.}}
\newcommand{\apjl}{{Astrophys. J.}}
\newcommand{\pasj}{{PASJ}}
\newcommand{\mnras}{{MNRAS}}
\newcommand{\apjs}{ApJS}
\newcommand{\araa}{ARA\&A}
\newcommand{\actaa}{Acta Astron.}
\newcommand{\na}{New Astronomy}
\newcommand{\nat}{Nature}
\newcommand{\memsai}{Memorie SAIT}
\newcommand{\ssr}{Space Science Reviews}
\newcommand{\pasp}{PASP}
\newcommand{\aapr}{A\&A~Rev.}
\newcommand{\physrep}{Phys.~Rep.}
\newcommand{\nar}{New A Rev.}%
\newcommand{\prd}{Phys.~Rev.~D}%
\newcommand{\jcap}{Journal of Cosmology and Astroparticle Physics}


\begin{document}

\titlepages                  % Please do not remove these lines
                             % Pages 1-4 are reserved for publisher
%\preface                     % requires preface.tex

\mastertoc                   % combined TOC showing all chapter titles

\setcounter{page}{1}

%\part{My Part Title}{}      % optional divider page

% Chapter 1
%\include{Theory/theory}

% Chapter 2
%\include{Colpi-Sesana/GW-universe-Colpi-Sesana}

% Chapter 3
%\include{JY-Vinet/vinet}

% Chapter 4
\chapter{Present and future ground-based detectors}

\author[Giacomo Ciani and Paul Fulda]{G.Ciani and P.Fulda\footnote{To be published in \textit{An overview of Gravitational Wave Theory and Detection.}, eds. G. Auger and E. Plagnol (World Scientific, 2016).}}
\address{University of Florida, Gainesville, FL 32611, USA}


\bigtoc

\tableofcontents

\body


\newpage
\begin{abstract}
	\textit{\textbf{Abstract}:
%		the direct detection of gravitational waves by the Advanced LIGO project on the centenary of their prediction by Einstein was a milestone event  for general relativity, gravitational physics, astrophysics and many other related fields.
%		Given the astonishingly small amplitude of the gravitational waves emitted by even the strongest sources, building an instrument sensitive enough to detect these tiny ripples in space-time was deemed impossible by Einstein himself.
%		Only the bright ideas of visionary physicists and many decades of intense scientific research and technological development have made the feat possible.
		In this chapter, we first briefly review the early history of gravitational wave detection, and how the research turned towards the large scale interferometers that proved to be the most effective devices for gravitational wave astronomy.
		We provide an overview of what is generically defined the ``first generation'' of interferometric detectors, a number of instruments built around the globe at the end of the last century, and differing in size, technology and scope.
		A more detailed description is dedicated to the ``second generation'', or ``advanced detectors'', and their main subsystems, which represent the state of the art of the science and technology in the filed.
		In the final section, we give a glimpse of what the next generation of detectors may look like.
		Given the many similitudes between different projects, throughout the chapter we will use LIGO as the leading example, and highlight how the other projects compared to it both technologically and strategically. This choice is mainly due to the size of the LIGO project and of the LIGO Scientific Collaboration, and to the pivotal role they played in the history of the field and eventually in the first detection of gravitational waves. It is important to note, however, that the comparatively small amount of space and detail dedicated to other endeavors is not indicative of the importance of the role they have played and still play in the field.
		}
\end{abstract}


\body


\newpage

\subsection{History of ground-based interferometric gravitational wave detectors}\label{subsec:prehistory}

The very earliest history of interferometric gravitational wave detectors lies in the experiments 
of the American physicists Albert Michelson and Edward Morley in the summer of 1887. Although 
the goal of the famous Michelson-Morley experiment was not to detect gravitational waves (indeed 
the theory that predicts their existence was still several decades away from being conceived), the basic 
design of their apparatus can still be found in every ground-based interferometric gravitational wave detector. 
Michelson and Morley used their interferometer to attempt to measure variations in the speed of 
light with direction of propagation, and as a result to measure the velocity of the Earth with respect to the 
luminiferous aether: a feature of the prevailing physical theories of the time. By the time of their 1887 measurement, 
their apparatus was deemed capable of measuring shifts of about 1\% of a fringe; a remarkable feat given the 
technology available. 

The negative results of the Michelson-Morley experiment eventually paved the way for Einstein's theory of special 
relativity, in which the speed of light is invariant with propagation direction. It is somehow fitting that a variant 
of the same apparatus was used in 2016 to make the first direct measurement of gravitational waves, themselves 
a key prediction of Einstein's theory of general relativity. 

%There is also an interesting parallel... somehow link MM experiment trying to find relative velocity to ether 
%using speed of light, Einstein relativity saying there is no such ether, GR saying GWs travel at speed of light, 
%MI proving twice that there is no relative velocity for light!

Efforts to detect gravitational waves began in earnest with Joseph Weber's development of resonant bar detectors 
in the 1960s. This detection scheme relied on the excitation of resonant modes of a mass with a high mechanical 
quality factor by passing gravitational waves. The reliance on resonances of the test mass produced a detector 
with an extremely limited bandwidth, able only even in principle to detect the presence of a GW signal, and not 
to uncover detailed information about the nature of the sources of the waves. 

Weber reported a series of detections throughout the 1960s. Efforts by Richard Garwin, Heinz Billing and others 
to reproduce his results were fruitless, however, and by the 1970s the veracity of Weber's detection claims was widely doubted. 
These early claims of detection and their subsequent dismissal are likely to be at least partly responsible for the 
overtly scrupulous nature of the modern gravitational wave detection field, evidenced for example in the great lengths to which 
the LVC collaboration went before publishing their first detection paper.

In the 1960s the idea of using laser interferometers as gravitational wave detectors was developed more or less simultaneously 
in several places, by Joseph Weber himself, along with soviet physicists Mikhail Gertsenshtein and Vladislav Pustovoit. 
It was not until 1968, however, that Rainer Weiss first performed a detailed noise analysis of a laser interferometer in the context 
of gravitational wave detection, considering all of the fundamental noise sources that still limit detectors to this day. 
It was this study that really demonstrated the feasibility of using large-scale laser interferometers for gravitational wave detection, and 
it was instrumental in securing funding for the further development of the technology, prototype interferometers, and eventually LIGO itself. 

The late 1960s through the early 1990s was the era of prototype interferometric gravitational wave detectors, beginning with 
Robert Forward (a former graduate student of Weber), through Weiss' prototype at MIT, the Garching prototype developed by 
Heinz Billing, and a prototype in Glasgow lead by Ronald Drever and James Hough. 
It was clear from Weiss' initial 
study that although reaching the required sensitivity to detect gravitational waves was possible in principle, a huge technological effort would 
be required to make that potential a reality. 
Prototype detectors were an essential part of that technology development. Funding a full-scale 
observatory was still deemed too risky during this era, and in any case the technology simply was not at a mature enough stage to give them 
a reasonable chance of detecting gravitational waves. These prototypes also provided the function 
of training young scientists in the methods and concepts that would be instrumental in 
designing, building and operating gravitational wave detectors. 

At the beginning of the 1980s the two projects that would eventually join together to form LIGO were initiated: 
a design study for a kilometer scale interferometer at MIT, and a 40m 
prototype interferometer at Caltech. Work continued on these projects, and the LIGO project persistently applied 
for funding through the late 1980s, eventually obtaining its full funding to the tune of nearly \$400 million in 1994. 

%Italian? French? Japan?


%\subsection{The first generation of interferometric detectors}\label{subsec:history}

In 1994, NSF approved funding for the construction of the two experimental facilities 
of the LIGO (Laser Interferometric Gravitational-wave Observatory) project, despite 
skepticism and some strong opposition from part of the physics and astronomy 
communities: many thought that the investment, the largest ever made by NSF on a 
single project, was too risky, that it would needlessly drain resources from other 
research, and that the chances of success would be almost non existent. History will prove them wrong.

Ground was broken the same year in Hanford, WA, and the following one in Livingston, LA; 
the construction of the buildings and of the vacuum system, by some measures the biggest 
ever built at the time, took almost five years. Separated by almost 3000 km, the two 
experimental sites shared the same basic design: a 4-km long L shape structure; they 
however differed for the orientation, and for the fact that the one in Hanford was 
designed to accommodate two parallel interferometers in the same vacuum system. 
The installation of the scientific equipment started in 1999 and was completed by 2002. 
In the meanwhile, a broader scientific community had grown around the LIGO project, 
and had taken the shape of two institutions: the LIGO laboratory, in charge of managing 
the facility and most of the research and development directly aimed at improving the 
instruments, and the LIGO Scientific Collaboration (LSC), formed by research groups 
around the world involved in technical and scientific research related to LIGO.

While construction of the two LIGO detectors was ongoing in the US, parallel efforts 
were being pursued in Europe. A French-Italian collaboration secured funding for a 
similar facility to be built in Cascina, near Pisa, in Italy. The construction of the 3-km 
long VIRGO interferometer started in 1996 and was completed in 2003. During this period, 
the European Gravitational Observatory (EGO) consortium was created to operate the detector 
and promote gravitational research in Europe. United Kingdom and Germany also joined 
forces to build the a large scale interferometer; the full-size project was not funded, but 
was de-scoped to a slightly smaller version named GEO600 (due to its 600 meters long arms), 
whose construction near Hannover, Germany, started in 1995.

Smaller scale interferometers, mainly intended as prototypes, were build or 
proposed in other parts of the world. In particular, ACIGO in Australia and 
CLIO and TAMA300 in Japan.

%TODO: things that can (must) be added, in loose order of importance
% Cuold expand a bit on LIGO inception based on LIGO Magazine piece (approved 1992, management issues, NSF continuing support, Barry Barish reboot, full steam ahead in 1994)
% names of major players
% name of institutions
% references
% detail of single and joint science runs of major detectors
% collaboration between LSC and VIRGO 

\subsection{Design and operation of the first generation of interferometric gravitational wave detectors}\label{subsec:1stgen}
In this and the following sections we describe the initial generation of ground based gravitational wave detectors and their subsequent major upgrade, often referred to as second generation. While this classification matches closely the upgrade history of the two largest scale interferometers, LIGO and Virgo, this may not necessarily be the case for other detectors that adopted different upgrade strategies or started development later. For these interferometers, the distinction that we make here between first and second - or even future - generations is to some degree arbitrary.


Despite the extensive research and development effort put forward and the experience acquired over several years on smaller prototypes, it was clear to the gravitational wave community and to the funding agencies that building and operation of km-scale interferometers at their design sensitivity was a high risk, although potentially high gain, endeavor.
The first generation of interferometers were designed to be relatively simple, reducing the odds of an actual detection but increasing the chances of their successful commissioning and operation.
In particular, while the adopted topology was very similar to the one used in the subsequent generation, the complexity of many subsystems were kept to a minimum.
The detector infrastructures constituted the bulk of the initial budgets however, and were built to be able to support future upgraded detectors

The initial LIGO detectors\cite{Abbott_2004,Abbott_2009} were power-recycled, Fabry-P\'{e}rot Michelson interferometers.
%REF: reference appropriate section
An out-of-vacuum, 10\,W, 1064\,nm Nd:YAG pre-stabilized laser was phase modulated to add three sets of sidebands for alignment and sensing control.
The beam was then spatially filtered and further stabilized in power and frequency by an in-vacuum input-optics section: this included a 24\,m round-trip suspended triangular optical cavity, referred to as input mode cleaner, a low loss Faraday isolator, and a suspended telescope to match the beam mode to that of the rest of the interferometer.
The beam was then injected into the recycling cavity,
%REF: reference appropriate section for "recycling cavity"
which increased the input power seen by the rest of the interferometer by a factor of about 50; finally, the power reached about 20 kW in the impedance matched arm cavities, designed to have a finesse of 220. There was no signal recycling cavity, and the strain signal was obtained using RF readout of the interferometer output.

The vacuum system layout, also designed to accommodate subsequent upgrades of the detectors, was based on a series of vacuum chambers of either of two types: horizontal ones, for the input and output optics, and vertical ones for the core optics.
Both type are cylinders about 2\,m in diameter and 3\,m in height, with large access ports for easy installation of heavy equipment.
The corner station at the Livingston observatory hosts 6 horizontal chambers, 3 on the input and 3  on the output branch, and 3 vertical chambers, with another two hosted in the end stations.
All chambers, including the ones at the end stations, are connected by 1.2\,m diameter vacuum tubes.
At Hanford, the need to accommodate a second interferometer required doubling the number of chambers, although most of the vacuum tubes were shared by the two laser beams and only minimal additions were needed to connect the extra chambers.
Each chamber was equipped with an optical table, passively isolated from seismic vibrations by a 4-stage spring-mass system, providing about 6 orders of magnitude isolation at 100\,Hz\cite{Giaime_1996}.
Single pendulum suspensions were used to support the most critical optical components, and in particular the 10\,Kg, 25\,cm diameter end mirrors of the Fabry-P\'{e}rot arm cavities, referred to as the input and output test masses; for these optics, the pendulum suspensions provided further 4 orders of magnitude suppression of ground motion at 100\,Hz.
In Livingston, where the ground motion is significantly higher than in Hanford, an out-of-vacuum hydraulic pre-isolation system was added to contribute another factor 10 suppression between 0.1 and 10\,Hz.

Before being decommissioned in 2010 to allow for the installation of the Advanced LIGO hardware, the LIGO detectors were fitted with a number of incremental upgrades\cite{Aasi_2015} meant to improve the sensitivity and allow for prototyping some of the technologies needed for the next generation of instruments.
The laser power was increased for 10\,W to 35\,W;
the thermal compensation system, which had been added to the initial detectors to correct for thermal lensing effects, was further improved to better handle the higher circulating power;
finally, a \textit{DC readout} detection scheme was implemented, in which the interferometer was operated with a slight offset from the dark fringe and the gravitational wave signal was read directly as a modulation of the power on the photodiode.
This required the installation of an output mode cleaner to filter out the RF control sidebands and higher-order spatial modes of the carrier light.
This version of the detectors is referred to as Enhanced LIGO (eLIGO), and conducted science operations in 2009 and 2010. 

The Virgo detector\cite{Accadia_2012}, in Italy, was based on a design similar to that of LIGO, except with 3\,km Fabry-P\'{e}rot arms cavities. 
The major difference in Virgo was the adoption of 7-stage, 10\,m tall three dimensional suspensions, named \textit{super-attenuators}, to isolate the input and output test masses.
A 3D model fo the superattenuator is depicted in \fref{fig:superattenuator}. The first stage is an inverted pendulum platform providing isolation at very low frequencies (about 30\,mHz) and actuation capabilities for coarse alignment and compensation of tidal effects.
From this platform hangs a chain of five cascaded single-wire pendula, each about 1\,m long; the mass of each pendulum, as well as the inverted pendulum stage, integrates a mechanical filter that provides vertical isolation for the suspension point of the subsequent stage; the vertical isolation is realized by supporting the suspension point with an array of pre-curved triangular steel blades which lay flat under the load and provide a vertical resonant frequency at about 1.5\,Hz.
The overall vertical resonant frequency is further lowered to below 0.5\,Hz by the adoption of magnetic anti-springs.
The payload is comprised of the test mass and a reaction mass, an hollow cylinder concentric with the test mass used a quite reference point for actuation on the mirror. Both are suspended to a crossbar, called the \textit{marionette}, via two loops of wire each.
The marionette is equipped with actuators that allow it, and consequently the payload, to be steered with respect to the above suspension stage. 
The \textit{super-attenuators} were designed to provide at least 10 orders of magnitude isolation down to 4 Hz, extending the Virgo observation band to lower frequencies compared to the other detectors of the same generation.
A shorter and simplified version of the superattenuator was used to suspend optical benches for less critical optics.

\begin{figure}
	\missingfigure[figwidth=12cm]{Superattenuator CAD drawing here?}
	\caption{\label{fig:superattenuator}A 3D model of the Virgo superattenuator}
\end{figure}

Similar to what was done with LIGO, Virgo was also equipped with a number of incremental upgrades aimed at improving the sensitivity and testing the maturity of technologies needed for the subsequent version of the detector.
Most notably, the laser power was increased from 10\,W to 25\,W, a thermal compensation system was added, and the test masses were suspended using fused silica fiber directly bonded to the optics to reduce thermal noise\cite{Lorenzini_2010}.
In this configuration, the instrument was referred to as Virgo+.
%REF: Specific reference for VIRGO+?

GEO600~\cite{Grote_2010}, on the other hand, adopted quite different design choices, and pioneered a number of innovative technologies of which several would later be integrated in the larger detectors:
instead of Faby-P\'{e}rot arm cavities it employed folded arms, a topology in which the end of the arms are occupied by folding mirrors that send the laser back towards the end test masses located close to the beam splitter;
it was the first detector to employ a signal recycling cavity to shape the gravitational wave signal frequency response~\cite{Willke_2002};
it also employed DC readout, rather than the more conventional RF homodyne readout~\cite{DCreadout};
it was the first detector to use monolithic final-stage suspensions of the test-masses~\cite{Plissi_2000};
finally, it was the first to employ squeezing, a technique used to shape quantum fluctuations and obtain a reduction in relative shot noise equivalent to that of a higher power laser~\cite{Grote_2013}.
GEO600 was also the only observatory to remain active during the period in which LIGO and VIRGO were being upgraded to their second generation.
%TODO: reference figure in the text
\begin{figure}[htb]
%	\includegraphics[scale=0.4\textwidth]{}
	\missingfigure[figwidth=12cm]{Schematics of GEO, LIGO, Virgo detectors to go here.}
	\caption{\label{fig:opticallayouts}Schematics optical layouts of GEO600, LIGO and Virgo.}
\end{figure}


TAMA300\cite{Ando_2002}, in Japan, also adopted an optical layout similar to LIGO and VIRGO, although its smaller size and location in Tokyo severely limited its sensitivity below a a few hundred Hz.
It was nevertheless instrumental in developing sensing and control techniques that would be later transferred to the larger interferometers. 
The CLIO detector, also in Japan but situated in the Kamioka mine, began construction in 2003 and was eventually operated with cryogenically cooled test masses and demonstrated a reduced thermal noise level from the room temperature~\cite{Uchiyama_2012}. However, with relatively short arm lengths of 100\,m CLIO could not come close to the strain sensitivities of the larger detectors.

In 2002, the LIGO detectors and GEO600 reached a sensitivity adequate for collecting science data, although still far from the design goal.
The first science run took place between August and September 2002 with a conventional range, intended as the maximum distance at which a standard NS-NS coalescence could be detected with a signal-to-noise ratio equal to 8, of 100 kpc, more than two orders of magnitude less than the design value of 18 Mpc.
In the years that followed, LIGO conducted other 5 science runs, interrupted by commissioning periods that steadily and consistently improved the performance until it finally reached full design sensitivity in 2006.
In all but one of these science runs, the three LIGO detectors were run in coincidence with one or more other large scale detectors around the world (GEO600, Virgo and TAMA300) to leverage the superior noise rejection and sky localization capabilities of a widely distributed network of detectors\cite{Abbott_2004,Abbott_2005,Abbott_2006,Abbott_2008,Abadie_2010}.

%NOTE: approximate list of runs (including partners):
%2002: S1 (LIGO + GEO)
%2003: S2 LIGO + TAMA DT8)
%2004: S3 (LIGO)
%2005: S4 (LIGO + GEO)
%2005-2007: S5 (LIGO + GEO + VIRGO VSR1)
%2009-2010: S6 + VSR2-3 

\section{Design and operation of the second generation of interferometric gravitational wave detectors}\label{subsec:2ndgen}

Even while the initial LIGO and Virgo detectors were still far from their design sensitivities, plans were afoot for major  upgrades to each, aimed at achieving roughly a factor 10 improvement in sensitivity over the whole frequency band. 
This generation of detectors would be known as the 2nd generation, or the advanced detectors; Advanced LIGO and 
Advanced Virgo. 

Besides generally targeting a factor 10 overall improvement in sensitivity, and an expansion of the sensitive 
frequency band towards lower frequencies, the design of the advanced detectors was aimed at making them 
limited by fundamental noises: thermal noise and quantum noise (a combination of quantum radiation pressure noise 
and laser shot noise).
As a consequence, all other possible sources of noise needed to be pushed well below these main ones.

\subsection{Advanced LIGO}
%TODO: reference in text
\begin{figure}[htb]
	\begin{center}
		\includegraphics[width=0.9\textwidth]{aLIGOnbudget_gray.pdf}
		\caption{\label{fig:aLIGOnbudget}Predicted strain equivalent spectral densities for major noise sources in Advanced LIGO at full input laser power. Also shown are the typical strain sensitivities of the Hanford and Livingston detectors throughout the O1 run, during which the first direct detection of gravitational waves was made.}
	\end{center}
\end{figure}

\subsubsection{Suspensions and seismic isolation.}
Two of the subsystems that were significantly upgraded from LIGO to Advanced LIGO
were the Seismic Isolation\cite{SEI2015} and Suspension subsystems. Although they 
are formally two separate subsystems, they work in concert to isolate the test-masses and 
other critical optics from ground vibrations and other macroscopic motions, and ensure 
that the they can move as free masses in the relevant degree of freedom above a few Hz.

In Advanced LIGO, the test masses are suspended by four-stage pendula, known as the quad suspensions\cite{Aston_2012}. 
From top to bottom, the main suspension chain is composed of two metal masses, and two 40\,kg cylindrical fused silica substrates
of 34\,cm diameter and 20\,cm thickness, the lowermost one being the test mass.
The first metal mass is attached to the suspension structure by four blade springs providing vertical isolation;
a steel wire runs from the tip of each blade to the 
suspended mass.
In a similar fashion, the second metal mass is attached to the first one, 
and so is the penultimate mass; itself actually a fused silica substrate of equal dimensions to the test masses themselves. 
In the final stage of the suspension, however, the test mass 
is attached to the penultimate mass by four fused silica fibers; these are directly welded to 
the two optics to form a monolithic assembly.
The monolithic suspension design offers reduced mechanical losses, and
consequently lower thermal noise, than the previously used metal wire suspensions.
The pendulum resonances of the four stages are 
distributed between 1 and 4 Hz, providing a passive suppression of motion along 
the optical axis by $10^7$ at 10\,Hz and improving as the frequency to the eighth power.

A similar chain of masses called the reaction chain hangs parallel to the main one, 
and supports the sensors and actuators used for local damping and active positioning and alignment control 
of the lowermost three masses of the main chain. The reaction chain provides a quiet reference point 
for the control forces.
The top masses of both chains are instead actuated using the suspension structure as a reference.
For all stages except the lowest, the sensors/actuators are compact units that use shadow sensors to measure the position of, and electromagnets to exert forces on, permanent magnets attached to the masses.
The last stage has no local sensors, since the position of the test mass is sensed 
by the global interferometry; on the end test mass, the actuators consist of patterns of electrodes 
deposited on the last mass of the reaction chain, which exert electrostatic forces on the test mass when polarized.
This avoids the need of attaching magnets to the test masses, thus maintaining low mechanical 
losses and reducing possible couplings to external fields.
The last stage of the reaction chain on the input test mass suspension has no actuators at all, and is instead used as the \textit{compensation plate} for the thermal compensation system.

Each quad suspension is attached to an in-vacuum seismic isolation platform, 
used for further suppression of ground vibrations and precise positioning and 
alignment with a larger range than allowed by the suspensions themselves. The 
platforms are six-axis, with two-stage active and passive isolators proving more 
than 3 orders of magnitude isolation above 1\,Hz, and positioning capabilities 
with nm resolution over a range of several mm.

Similar seismic isolation platforms are used to support all the in-vacuum optics; 
the optics that are part of an optical cavity are further suspended by triple-pendula, while 
single-pendulum suspensions, or specialized geometries, are used to isolate 
less critical optics where necessary.

In both the Hanford and Livingston Advanced LIGO detectors, 
each of the in-vacuum seismic isolation platforms is 
installed on beams that are decoupled from the vacuum system itself via 
flexible bellows, and supported from the outside using hydraulic actuated 
piers anchored to ground.
This system acts as a further layer of isolation and is used to predictively correct for macroscopic positioning drifts caused by tidal forces from the moon and the sun.

\subsubsection{Laser}
At frequencies above about 100\,Hz, the interferometer sensitivity is limited by shot 
noise in the laser. The relative impact of the shot noise scales as the inverse of the 
square root of the power: increasing the laser power is thus a conceptually 
straightforward way of improving the sensitivity in the shot-noise limited band. 
The Advanced LIGO laser source is designed to deliver a maximum of 180\,W of laser power, as opposed to the 35\,W used in eLIGO.
In order to achieve this goal the laser source is composed of a 2\,W Nd:YAG 1064\,nm non-planar ring oscillator (NPRO) master laser, amplified up to 35\,W by a single-pass medium-power amplifier, subsequently amplified to 220\,W by an injection-locked ring ring oscillator known as the high-power oscillator stage~\cite{Kwee_2012}. 

This beam is then pre-stabilized in frequency with respect to a fixed spacer cavity in a thermally shielded environment, and pre-stabilized in intensity with respect to several reference photodiodes. 
The beam from the pre-stabilized laser is also passed through a pre-mode cleaner ring cavity, which filters the spatial mode of the laser ensuring a high-purity Gaussian beam profile. 
The beam is then handed off to the Input Optics subsystem, where phase modulation sidebands are applied and the power of the beam is controlled, before the beam is passed to the in-vacuum suspended input mode cleaner cavity~\cite{Mueller_2016}. 
This cavity serves to further filter the beam in both frequency and spatial mode, passively suppressing any beam jitter of the pre-stabilized laser beam from non-isolated optical components. 
The beam transmitted from the input mode cleaner is then passed through a Faraday isolator, before being expanded and matched to the 
main interferometer mode.

\subsubsection{Thermal compensation system}
Despite the stringent requirements on the optical absorption of bulk and coating 
material of the optics, the high power levels circulating in the interferometer result 
in a non-negligible amount of heat deposited into the optics.
Due to the poor heat conduction in vacuum, this induces important thermal gradients that can modify 
the optical parameters of the system via two main effects: thermal lensing in the 
bulk material due to the temperature dependence of the refractive index,
and distortion of the high reflectivity (HR) surface of the mirrors due to thermo-mechanical stress.
HR surface distortion is particularly important for the input and end test masses, both because they see the highest power level of all the interferometer optics (up to about 1\,MW), and because any deformation of their surface has a bigger impact on the interferometer output.
Thermal lensing, while irrelevant for the end test mass due to the secondary role of the weak transmitted diagnostic beam, is an important effect in the input test masses, since it can spoil both the mode matching with the power recycling cavity and the mode overlap between the two arm cavities, thus decreasing the overall contrast of the interferometer.

The Advanced LIGO thermal compensation system is designed to monitor and compensate for both effects across the entire range of operating powers, and constitutes a substantial improvement over the much simpler implementation used in eLIGO.
To sense the thermal distortion, each of the four input and end test masses is monitored using a custom Hartman wavefront sensors, which uses an auxiliary superluminescent diode beam injected from the anti-reflection face of the optic and reflected back from the HR side (thus traversing the optic twice).
To correct for HR surface distortions, an infrared annular heater heats the barrels of the test masses, reducing the thermal gradient and inducing a thermal stress that counteracts the effect of the central heating due to the main laser beam.
Finally, a CO$_2$ laser projector is used to impress a suitable pattern on the compensation plate, deliberately creating a thermal lens that compensates for the lens left in the input test mass after the combined effects of the science beam and the annular heater~\cite{TCS_inpreparation}.

\subsubsection{Optical layout}
The optical layout of Advanced LIGO is different from the initial LIGO layout in several ways. 
The most fundamental change to the optical layout is the addition of a signal recycling 
mirror between the anti-symmetric side of the beam splitter and the optical detection port. 
In its current configuration this oft-called signal recycling mirror is actually tuned such as to 
increase the bandwidth of the detector, rather than increasing the quantum noise-limited sensitivity
in a narrow band as the name \textit{recycling} implies. As such, a more apposite name for this mirror 
in the current configuration is signal \textit{extraction} mirror. 

The signal extraction mirror forms a new cavity within Advanced LIGO; which is still called the signal recycling cavity. 
Both the signal recycling cavity and the power recycling cavity in Advanced LIGO are designed to be 
geometrically \emph{stable}, by which it should be understood that the round-trip Gouy phase in the cavity 
is significant, and thus higher-order spatial modes are non-degenerate~\cite{Arain2008}. 
This is in contrast to the power recycling cavity in initial LIGO, which was only marginally stable. 
The advantages of the stable recycling cavity design have been clear during the commissioning of 
Advanced LIGO, where commissioning of the length and alignment sensing and control systems has 
been a much smoother process than in initial LIGO. 

Another important geometric change to the optical layout between initial LIGO and Advanced LIGO is 
in the arm cavities. There was a drive towards using larger beam spot sizes on the mirrors in Advanced 
LIGO in order to mitigate the effects of thermal noise. In general there are two cavity geometry solutions available 
that will give a specific beam spot size on the mirrors for a two-mirror cavity of fixed length. The initial LIGO 
arm cavities were designed with a large beam waist inside the cavities
whereas Advanced LIGO uses the alternative solution of having a small beam waist size in the cavities. 
The major advantage to the small beam waist size solution is that thermal deformations of the test masses caused by 
absorption in the coatings push the cavity to a more stable geometry, rather than towards a less stable geometry as 
is the case for the large beam waist design. 

Several additional optical subsystems have been added in the upgrade from initial LIGO to Advanced LIGO. 
During the enhanced LIGO phase (shortly before initial LIGO went offline for the major upgrade to aLIGO) an output
mode cleaner was added at the output port. The output mode cleaner is a crucial component of the DC readout 
scheme which was first demonstrated in GEO600, and which was determined to be a more optimal solution for 
readout of the gravitational wave signal than the previously used RF heterodyne readout scheme~\cite{DCreadout}. The output mode cleaner subsystem was retained in the aLIGO optical layout, and takes the form of a suspended fixed spacer cavity with a bow tie configuration.
The output mode cleaner has the essential function of removing RF sidebands and higher-order spatial modes from the light incident on the photodiode, thus mitigating their impact on the shot noise sensitivity. 

A great effort was made in the upgrade to Advanced LIGO to make the lock acquisition process more deterministic than stochastic.
Part of this effort was the inclusion of the arm length stabilization (ALS) subsystem.
This subsystem uses green frequency-doubled Nd:YAG beams which are phased locked to the main laser to independently control the arm cavities during lock acquisition of the central dual-recycled Michelson interferometer (DRMI)~\cite{Staley2014}.
Once the DRMI is locked the ALS can be used to methodically bring the arms to resonance, bringing the full interferometer to the ideal operating point. 
The arm length sensing can be handed off to the main interferometric sensors, once the ALS has brought the interferometer within their linear range.

\subsubsection{Interferometric sensing and control}
The dual-recycled Fabry-P\'{e}rot Michelson interferometer that makes up aLIGO has a very narrow linear range. 
The practical consequence of this fact, when combined with the fact that even with the advanced seismic isolation systems typical 
mirror motions at low frequencies can be of the order several wavelengths, is that length control loops are essential in order 
to keep the interferometer operating with high sensitivity. 

The length sensing of all interferometric ground-based gravitational wave detectors is based on the Pound-Drever-Hall (PDH) laser 
frequency stabilization scheme~\cite{PDH}. 
In this scheme an electro-optic modulator is used to add RF phase modulation sidebands to the laser, 
which are typically non-resonant in an optical cavity when the carrier light is resonant.
When the main light frequency, typically called the carrier frequency, is brought close to resonance in the cavity, 
it picks up a phase shift in reflection of the cavity which is proportional to the difference between its 
frequency and the cavity resonant frequency. The sidebands act as a phase reference for comparison with the 
carrier light, and the total reflected light 
from the cavity is detected with a photodetector and demodulated at the original modulation frequency to give an error 
signal for either the cavity length control or the laser frequency control. 

While the PDH scheme described above refers to the sensing of just one length (or frequency) degree of freedom, 
the core interferometers of 2nd generation ground-based gravitational wave detectors require 5 distinct length degrees of freedom to 
be sensed and controlled. Typically these degrees of freedom are broken down into the following list: common arm length, 
differential arm length (where the gravitational wave signal predominantly appears), Michelson tuning, power recycling cavity length and 
signal recycling cavity length. In reality at least two additional degrees of freedom must be controlled; one each for the input 
and output mode cleaner cavities. This presents a formidable challenge, which was solved for Advanced LIGO by the use of two 
different modulation frequencies, at roughly 9\,MHz and 45\,MHz. The 9\,MHz sidebands are resonant in the power recycling 
cavity only, and experience a dark Michelson fringe. Detectors at various ports demodulated at this frequency typically provide 
good length sensing signals for the arm degrees of freedom, as well as the power recycling cavity length. The 45\,MHz sidebands 
are resonant in both power and signal recycling cavities, and experience a bright Michelson fringe. As a result, the detectors demodulated 
at 45\,MHz provide good sensitivity to signal recycling cavity length and the small Michelson degrees of freedom.

The alignment of optics must also be sensed and controlled in gravitational wave detectors. In aLIGO the sensing is currently achieved using a method 
called differential wavefront sensing, developed by Henry Ward and colleagues~\cite{Morrison1994, Morrison1994b}. 
This method is similar in principle to the PDH length sensing, 
except that quadrant photodectors are used 
instead of single-element photodetectors in order to measure the beats between sidebands and carrier in different spatial modes. An 
alternative method developed by Dana Anderson was used in Virgo, whereby the sideband frequencies were chosen such that higher-order 
spatial modes of the sidebands would be co-resonant in some of the optical cavities with the carrier fundamental mode~\cite{Anderson1984}. 
A detailed description of the alignment requirements of a gravitational wave detector, and a sensing scheme based on the wavefront sensing 
is provided in~\cite{Fritschel1998}.

\subsection{Advanced Virgo}
%ISSUE: I think we need to expand this... or take it out alltogether
At the time of writing, Advanced Virgo is still in the process of installing upgrades and commissioning. 
The superattenuator test-mass suspensions in Virgo already performed extremely well for suppressing the coupling of ground motion to test-mass motion, and so upgrades in this area were minimal.

The laser power is being increased in order to improve the shot noise sensitivity of the detector, and new test-masses with low mechanical loss coatings are being installed.

The optical layout is very similar to Advanced LIGO, being a dual-recycled Fabry-P\'{e}rot Michelson interferometer employing DC readout with an output mode cleaner.
Due to restrictions in available vacuum enclosure space, however, stable recycling cavities were not a feasible design option for Advanced Virgo. 


%\item Sesmic isolation upgrade?
%\item Laser power upgrade
%\item Optical layout
%\item sensing and control
%\item Thermal compensation
%\item signal recyling 

\subsection{The future of ground based interferometric gravitational wave detection}\label{subsec:future}

\begin{itemize}
	\item Near/mid future
	\begin{itemize}
		\item NIGO India
		\item In-situ upgrades to LIGO
		\item Kagra
		\item Virgo? Not sure they have anything in mind before ET...
	\end{itemize}
	\item Long term
	\begin{itemize}
		\item LUNGO (or similar)
		\item ET
	\end{itemize}
\end{itemize}
 

\bibliographystyle{ws-rv-van}    % Bibliography: Author-Date system
\bibliography{presentandfuture}      % pls. call your database here

% Chapter 5
%\include{Lisa/Guido}

% Chapter 6
%\include{LPF/Rita}

% Chapter 7
%\include{PTA/Petiteau_PTA}

% Chapter 8
%\include{Geiger/geiger}


\printindex[aindx]           % to print author index
\printindex                  % to print subject index

\end{document} 